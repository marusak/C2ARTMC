% Copyright 2004 by Till Tantau <tantau@users.sourceforge.net>.
%
% In principle, this file can be redistributed and/or modified under
% the terms of the GNU Public License, version 2.
%
% However, this file is supposed to be a template to be modified
% for your own needs. For this reason, if you use this file as a
% template and not specifically distribute it as part of a another
% package/program, I grant the extra permission to freely copy and
% modify this file as you see fit and even to delete this copyright
% notice. 


\documentclass[slovak]{beamer}
\usepackage[slovak]{babel}
\usepackage[utf8]{inputenc}

% There are many different themes available for Beamer. A comprehensive
% list with examples is given here:
% http://deic.uab.es/~iblanes/beamer_gallery/index_by_theme.html
% You can uncomment the themes below if you would like to use a different
% one:
%\usetheme{AnnArbor}
%\usetheme{Antibes}
%\usetheme{Bergen}
%\usetheme{Berkeley}
%\usetheme{Berlin}
%\usetheme{Boadilla}
%\usetheme{boxes}
%\usetheme{CambridgeUS}
%\usetheme{Copenhagen}
%\usetheme{Darmstadt}
%\usetheme{default}
%\usetheme{Frankfurt}
%\usetheme{Goettingen}
%\usetheme{Hannover}
%\usetheme{Ilmenau}
%\usetheme{JuanLesPins}
%\usetheme{Luebeck}
\usetheme{Madrid}
%\usetheme{Malmoe}
%\usetheme{Marburg}
%\usetheme{Montpellier}
%\usetheme{PaloAlto}
%\usetheme{Pittsburgh}
%\usetheme{Rochester}
%\usetheme{Singapore}
%\usetheme{Szeged}
%\usetheme{Warsaw}

\title{Prekladač z fragmentu jazyka C do ARTMC}

% A subtitle is optional and this may be deleted
\subtitle{Obhajoba semestrálneho projektu}

\author{M.Marušák}
% - Give the names in the same order as the appear in the paper.
% - Use the \inst{?} command only if the authors have different
%   affiliation.

\institute[Vysoké učení technické] % (optional, but mostly needed)
{
  Ústav inteligentých systémov\\
  Vysoké učení technické v Brne

}

\date{ISP, 2017}
% - Either use conference name or its abbreviation.
% - Not really informative to the audience, more for people (including
%   yourself) who are reading the slides online


% If you have a file called "university-logo-filename.xxx", where xxx
% is a graphic format that can be processed by latex or pdflatex,
% resp., then you can add a logo as follows:

% \pgfdeclareimage[height=0.5cm]{university-logo}{university-logo-filename}
% \logo{\pgfuseimage{university-logo}}

% Delete this, if you do not want the table of contents to pop up at
% the beginning of each subsection:
\AtBeginSubsection[]
{
  \begin{frame}<beamer>{Obsah}
    \tableofcontents[currentsection,currentsubsection]
  \end{frame}
}

% Let's get started
\begin{document}

\begin{frame}
  \titlepage
\end{frame}

\begin{frame}{Úvod}
  \tableofcontents
  % You might wish to add the option [pausesections]
\end{frame}

% Section and subsections will appear in the presentation overview
% and table of contents.
\section{Zadanie}


\begin{frame}{Zadanie}
    \large\textbf{Prekladač z fragmentu jazyka C do nástroja ARTMC}\\
    \small\textbf{Compiler of C Language Fragment to ARTMC Tool}\\
    \normalsize\textbf{Vedúci:} Rogalewicz Adam, doc. Mgr., Ph.D.\\
    \normalsize\textbf{Zadanie:}
    \begin{enumerate}
        \item Naštudujte vstupný formát nástroja ARTMC
        \item Vyberte vhodnú podmnožinu jazyka C s ohľadom na možnosti nástroja ARTMC
        \item Navrhnite spôsob prekladu vybranej podmnožiny jazyka C do vstupného formátu nástroja ARTMC
        \noindent\makebox[\linewidth]{\rule{\paperwidth}{0.4pt}}
        \item Navrhnutý prekladač implementuje
        \item Výsledný prekladač otestuje na príkladoch z distribúcie nástroja ARTMC a ďalej na sade 10 nových demonštračných príkladov.
        \item Diskutuje možné pokračovanie a rozšírenie tohoto projektu
    \end{enumerate}
    \normalsize\textbf{Implementačný jazyk:} Python\\
\end{frame}

\section{Nástroj ARTMC}

\subsection{Čo je to ARTMC}
\begin{frame}{Nástroj ARTMC}
\begin{block}{Čo je to ARTMC}
ARTMC je nástroj na verifikáciu programov pracujúcich nad dynamickými dátovými štruktúrami. Cieľom nástroja je overenie, že v programe nemôže nastať nedovolená operácia. (napr. práca s Null ukazateľom)\\
Používa metódu Abstraktného Regulárneho stromového Model Checkingu
\end{block}
\end{frame}

\subsection{Vstup jazyka ARTMC}
\begin{frame}{Vstup jazyka ARTMC}
\begin{block}{\texttt{program.py}}
\textbf{Súbor obsahujúci definíciu programu vo vstupnom formáte}\\
Jedna python funkcia vracajúca dvojicu (program, env)\\
\textit{Viac na ďaľšom slide}
\end{block}
\begin{block}{\texttt{typedefs}}
\textbf{Súbor obsahujúci definíciu počiatočnej konfigurácie}\\
http://www.fit.vutbr.cz/research/groups/verifit/tools/artmc/DOC-typedefs.txt
\end{block}
\end{frame}


\subsubsection{(program, env)}
\begin{frame}{(program, env)}
\begin{block}{\texttt{program}}
\textbf{Zoznam n-tíc - každá n-tica jeden príkaz programu}\\
Príkazov je 15\\
Napr:\\
x=null	\rightarrow			("x=null","line\_num",x,next\_line)\\
x.next=null	\rightarrow		("x.next=null","line\_num",x,next,next\_line)\\
goto	\rightarrow			("goto","line\_num",next\_line)\\
if x==y	 \rightarrow	
(''if x==y","line\_num",x,y,next\_line\_then,next\_line\_else)
\end{block}
\begin{block}{\texttt{env}}
\textbf{N-tica v tvare (node\_width,pointer\_num,desc\_num,next\_num,err\_line,restrict\_var)}
\end{block}

\begin{alertblock}{}
Absolútne adresovanie, zložitý zápis, obtiažna kontrola správnosti, takmer nečitateľné
\end{alertblock}
\end{frame}

% Placing a * after \section means it will not show in the
% outline or table of contents.
\section{Práca na projekte}

\subsection{Progres za zimný semester}
\begin{frame}{Progres za zimný semester}
  \begin{itemize}
  \item Štúdium vstupu ARTMC, vedieť vytvoriť program ručne
  \item Výber podmnožiny jazyka C
  \item Lexikálna analýza - scanner 
  \item Syntaktická analýza zhora-nadol - parser - kostra
  \item Automatické testy - pripravené + Tox
  \item Dokumentácia - WIP
  \end{itemize}
\end{frame}

\subsection{Plán na letný semester}
\begin{frame}{Plán na letný semester}
  \begin{itemize}
  \item Dokončiť parser
  \item Napísať testy
  \item Dokončiť dokumentáciu
  \end{itemize}
\end{frame}

\end{document}


