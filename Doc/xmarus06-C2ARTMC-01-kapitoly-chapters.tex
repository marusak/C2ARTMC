%=========================================================================
% (c) Michal Bidlo, Bohuslav Křena, 2008

\chapter{Úvod}

\chapter{Nástroj ARTMC}
ARTMC je nástroj na formálnu verifikáciu programov, ktoré manipulujú s
 dynamicky viazanými dátovými štruktúrami. Tento nástroj bol vyvinutý na Fakulte
 Informatiky Vysokého Učení Technického v Brne vrámci výskmnej skupiny VeriFit,
 ktorá sa zaoberá verifikáciou programov. V tejto kapitole je tento nástroj
 popísaný z dvoch hľadísk. Najskôr je stručný prehľad funkcionality jazyka a
 následne je popísaná štruktúra vstupného súboru, ktorý je pre túto prácu
 najdôzležitejší.
-- CO je to verifikacia?
-- Co su to dynamicke datove struktury?
 \section{Funkcionalita a použitie ARTMC}
 Majme na vstupe nerekurzívny program, ktorý manipuluje s dynamicky viazanými
 šturktúrami s viacnásobnými next ukazateľmi. Snahou nástroja je overiť, že
 nemôžu nastať nedovolené operácie (napr. zápis do null ukazateľa, použitie
 nedefinovaného alebo už zmazaného prvku atď.).

 Verifikačná metóda, ktorá je použitá v tomto nástroji je založená na anstraktom
 regulárnom stromovom model checkingu. MORE 

 \section{Formát vstupného súboru}
 Na vstupu musia byť dva súbory:
 1. Typedefs - SOMETHING
 2. Zdrojový súbor obsahujúci jednu funkciu v jazyku Python, ktorá
 vracia dvojicu \texttt{(program, env)}.
 Prvý prvok \textit{program} je zoznam n-tíc, kde každá ntica je jeden príkaz
 jazyka ARTMC. V tabuľke \ref{table:prikazy} sú popísané jednotlivé podporované
 konštrukcie.
 \begin{table}[]
 \begin{tabular}{ll}
 x:=null             &          ("x=null","line\_num",x,next\_line)\\
 x:=y                &  ("x=y","line\_num",x,y,next\_line)\\
 x:=y.next           & ("x=y.next","line\_num",x,y,next,next\_line)\\
 x.next=y            &("x.next=y","line\_num",x,y,next,next\_line,descr\_num)\\
 if x==NULL          &  ("ifx==null",line\_num",x,next\_line\_then,next\_line\_else)\\
 if x==y             &     ("ifx==y","line\_num",x,y,next\_line\_then,next\_line\_else)\\
 if *                & ("if*","line\_num",next\_line\_then,next\_line\_else)\\
 goto                & ("goto","line\_num",next\_line)\\
 exit                &   ("exit","line\_num")\\
 x.next=null         &      ("x.next=null","line\_num",x,next,next\_line)\\
 xnext=new           &  ("x.next=new","line\_num",x,next,next\_line, descr\_num, gen\_descr)\\
 setdata             &     ("setdata","line\_num",x,"data",next\_line)\\
 if x.data=="..."    &            ("ifdata","line\_num",x,"data",next\_line\_then,next\_line\_else)\\
 x:=random\_position  &           ("x=random","line\_num",x,next\_line)\\
 new                  &   ("new","line\_num",x,next\_line)\\
 \end{tabular}[]
 \caption{Podporované príkazy v ARTMC}
 \label{table:prikazy}
 \end{table}
 \subsection{Popis príkazu jazyka pre ARTMC}
 \begin{itemize}
     \item[Identifikátor] Prvá položka je vždy reťazec obsahujúci identifikátor inštrukcie.
     \item[Číslo príkazu] Druhá položka značená ako \textit{line\_num} identifikuje príkaz
         vrámci celého programu. Jedná sa o reťazec skaldajúci sa z postupnosti
         núl a jedničiek, ktorý musí byť vrámci celého  programu jedinečný.
         Ideálny postup je číslovať inštukcie od 0 až po N a previesť tieto
         čísla do binárnej podoby.
     \item[Premenné] Označené ako \textit{x} a \textit{b}. V programe sú značené prirodzenými číslami.
     \item[Ukazateľové premenné] Označené ako \textit{next}. V programe sú číslované od 0.
     \item[Ukazateľ na dalšiu inštrukciu] Označené ako \textit{next\_line}. Označuje
         číslo riadku inštrukcie, na ktorej sa po skončení bude pokračovať. Inštrukcie
         sú číslované od nuly. Ak je použítá prípona \textit{\_then} alebo \textit{\_else}
         jedná sa o miesto pokračovania v prípade splnenia alebo nesplnenia podmienky danej inštrukcie.
     \item[Deskriptory 1] dunno now
     \item[Desktirpty 2] dunno now
 \end{itemize}
\section{Tvorba programu v ARTMC}
Preco pisanie v ARTMC je pain a ako v C nie je a prelozenie a ze uz je aj VERLIS (odkaz na akpiotlu)

\chapter{Prekladače}
Prekladač je nástroj, ktorý vytvára kód z jazyka vyššej úrovne do jazyka nižsej úrovne.
MORE
\section{Analýza zhora-nadol}
Ako to ide, vyhody, kde sa to pozuiva
\section{Analýza zdola-nahor}
Ako to ide, vyhody, kde sa to pozuiva
\section{LL-gramatika}
Co to je, ako sa tvori + v prilohe moja
\chapter{Súčasný stav prekladača pre ARTMC}
bola snaha, co to je...
\section{AVERILES}
kto, preco v com a ako
\chapter{Implementácia}
Táto kapitola sa venuje navrhovaniu implementácie...
\section{Python}
\section{Git}
\section{Tox}
\section{Unittest}
\section{OO}
\chapter{Výsledná aplikácia}
\section{Testy}
\section{Nedostatky}
\section{Navrhovaná práca}
\chapter{Záver}

%=========================================================================
